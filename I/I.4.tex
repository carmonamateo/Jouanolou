%%%%%%%%%%%%%%%%%%%%%%%%%%%%%%%%%%%%
\subsection*{4. Opérations externes.}
\addcontentsline{toc}{subsection}{4. Opérations externes}

On suppose donné dans ce paragraphe un anneau commutatif unifère $A$ et un idéal propre $J$ de $A$.

\vskip .3cm
{\bf 4.1}. Soient $X$ et $Y$ deux topos et $f: X \to Y$ un morphisme de topos. Ayant choisi un foncteur image réciproque
$$
f^*: A-\Mod_Y \to A-\Mod_X,
$$
on définit un foncteur exact, noté de même,
$$
f^*: \mathcal{E}(Y, J) \to \mathcal{E}(X, J)
$$
en posant pour tout $A$-faisceau $F = (F_n)_{n \in \mathbf{N}}$ sur $Y$
$$
f^*(F) = (f^*(F_n))_{n \in \mathbf{N}}
$$
et pour tout $\mathcal{E}(Y, J)$-morphisme $u = (u_n)_{n \in mathbf
N}$,
$$
f^*(u) = (f^*(u_n))_{n \in \mathbf{N}}.
$$
Ce foncteur est exact et transforme évidemment $A$-faisceau négligeable en $A$-faisceau négligeable. On en déduit par passage au quotient un foncteur exact, appelé foncteur \emph{image réciproque par $f$}, 
$$
f^*: A-\fsc(Y) \to A-\fsc(X).
\leqno{(4.1.1)}
$$
Il est clair que deux foncteurs images réciproques de $A-\Mod_Y$ dans $A-\Mod_X$, étant isomorphes, définissent des foncteurs isomorphes de $A-\fsc(Y)$ dans $A-\fsc(X)$ ; par suite, on pourra parler, sans plus d'ambiguïté que dans le cas des faisceaux de $A$--Modules, ``du'' foncteur image réciproque. 
\vskip .3cm
{\bf Exemple 4.1.2}. Le foncteur restriction (1.6.2) associé à un morphisme $f: T \to T'$ d'objets d'un topos $X$ n'est autre que le foncteur image réciproque associé au morphisme de topos
$$
X/T \to X/T'
$$
correspondant.
\vskip .3cm
{\bf 4.1.3}. Si maintenant $f: X \to Y$ et $g: Y \to Z$ sont deux morphismes de topos, on définit sans peine, argument par argument, un isomorphisme de foncteurs
$$
(g \circ f)^* \isom f^* \circ g^*,
$$
vérifiant la condition de cocycles habituelle.
\vskip .3cm
{\bf 4.1.4}. Notant ``$\pt$'' le topos ponctuel, i.e. la catégorie des ensembles munie de la topologie canonique, on rappelle (2.8.1) que la catégorie $A-\fsc(\pt)$ s'identifie à la sous-catégorie pleine de $\Pro(A-\mod)$ engendrée par les systèmes projectifs $M = (M_n)_{n \in \mathbf{N}}$ de $A$--modules vérifiant $J^{n+1}M_n = 0$ pour tout $n \geq 0$. Si maintenant $X$ est un topos et 
$$
p: X \to \pt 
$$
le morphisme de topos canonique, le foncteur $p^*$ associe à tout système projectif $M$ comme ci-dessus un $A$-faisceau sur $X$, qui sera noté de même s'il n'y a pas de confusion possible.

Supposons maintenant que $A$ soit \emph{noethérien}. Il résulte du lemme d'Artin-Rees (EGA $0_I$ 7.3.2.1). que le foncteur
$$
M \mapsto (M/J^{n+1}M)_{n \in \mathbf{N}}
\leqno{(4.1.4.1)}
$$
de la catégorie $A-\modn$ des $A$--modules de type fini dans $A-\fsc(\pt)$ est exact et fidèle,et même pleinement fidèle lorsque $A$ est complet pour la topologie $J$-adique (EGA $0_I$ 7.8.2). Composant avec le foncteur $p^*$, on en déduit un foncteur exact et fidèle
$$
A-\modn \to A-\fsc(\pt),
\leqno{(4.1.4.2)}
$$
qui est de même pleinement fidèle lorsque $A$ est complet pour la topologie $J$-adique. Dans la suite, on identifiera si aucune confusion n'en résulte un $A$--module de type fini et le système projectif n'en résulte un $A$--module de type fini et le système projectif associé au moyen du foncteur (4.1.4.2). 
\vskip .3cm
{\bf 4.1.5}. Soit $X$ un topos. Tout foncteur point
$$
i: \pt \to X
$$
de $X$ définit un foncteur exact
$$
i^*: A-\fsc(X) \to A-\fsc(\pt) \hookrightarrow \Pro(A-\mod),
$$
qu'on appellera \emph{foncteur fibre associé à $i$}. On prendra garde que si $X$ admet une famille conservative $(i_r)_{r \in R}$ de foncteurs points, alors la famille de foncteurs exacts
$$
i^*_r: A-\fsc(X) \to A-\fsc(\pt)
$$
n'est pas en général conservative. Soient en effet $X$ un espace topologique quasicompact et $(x_r)_{r \in \mathbf{N}}$ une infinité dénombrable de points fermés de $X$, et notons $i_r: x_r \to X$ les inclusions canoniques. On peut choisir une infinité dénombrable $(P_r)_{r \in \mathbf{N}}$ de $A$-faisceaux essentiellement nuls sur le   topos ponctuel, telle qu'il n'existe aucune application croissante $\gamma \geq \id: \mathbf{N} \to \mathbf{N}$ pour laquelle les morphismes canoniques $\chi_\gamma(P_r) \to P_r$ soient simultanément nuls. Alors le système projectif de $A_X$--Modules
$$
P = \bigoplus_{r \in \mathbf{N}}(i_r)_* (P_r)
$$
n'est pas essentiellement nul et définit pourtant un $A$-faisceau qui est envoyé sur le $A$-faisceau nul par tous les foncteurs fibres associés aux points de $X$. Nous verrons cependant plus loin que ce genre d'inconvénient ne se produit plus lorsqu'on fait des hypothèse de finitude convenables sur les $A$-faisceaux envisagés.
\vskip .3cm
{\bf 4.2}. Soient $X$ et $Y$ deux topos et $f: X \to Y$ un morphisme. Avec les abus de langage usuels, on définit pour tout entier $p$ un foncteur additif
$$
\Rd^pf_*: \mathscr{E}(X, J) \to \mathscr{E}(Y, J)
$$
par la formule
$$
\Rd^p f_* ((F_n)_{n \in \mathbf{Z}}) = (\Rd^p f_* (F_n))_{n \in \mathbf{N}}.
$$
Mieux, la collection des foncteurs $(\Rd^p f_*)_{p \in \mathbf{Z}}$ est munie de fa\c{c}on évidente d'une structure de foncteur cohomologique de $\mathscr{E}(X, J)$ dans $\mathscr{E}(Y, J)$.
\vskip .3cm
{
Lemme {\bf 4.2.1}. --- \it On suppose $f$ \emph{quasicompact} (SGA4 VI 3.1). Soit $u: F \to G$ un morphisme de $\mathscr{E}(X, J)$ dont le noyau et le conoyau sont négligeables. Alors pour tout entier $p \in \mathbf{Z}$, le morphisme 
$$
\Rd^p f_* (u): \Rd^p f_* (F) \to \Rd^p f_* (G)
$$
est à noyau et conoyau négligeables.
}
\vskip .3cm
{\bf Preuve} : Comme $f$ est quasicompact, on voit en se ramenant au cas où l'objet final de $X$ est quasicompact que pour tout $q \in \mathbf{Z}$ le foncteur $\Rd^q f_*$ transforme $A$-faisceau négligeant en $A$-faisceaux négligeable. Le lemme s'en déduit en utilisant la structure de foncteur cohomologique sur $(\Rd^p f_*)_{p \in \mathbf{N}}$.

D'après les propriétés générales des catégories abéliennes quotients, la catégorie $A-\fsc(X)$ est obtenue à partir de $\mathcal{E}(X, J)$ en inversant les flèches dont le noyau et le conoyau sont négligeables. Il résulte de (4.2.1) que lorsque $f$ est \emph{quasicompact} le foncteur cohomologique $(\Rd^p f_*)_{p \in \mathbf{Z}}$ définit par passage au quotient un foncteur cohomologique noté de même
$$
(\Rd^p f_*)_{p \in \mathbf{Z}}: A-\fsc(X) \to A-\fsc(Y).
\leqno{(4.2.2)}
$$
Bien entendu, $\Rd^i f_* = 0$ pour $i < 0$, et en particulier le foncteur \emph{image directe} $f_* = \Rd^0 f_*$ est exact à gauche.
\vskip .3cm
{
Définition {\bf 4.2.3}. --- \it Soit $X$ un topos. On dit qu'un $A$-faisceau $F = (F_n)_{n \in \mathbf{N}}$ est \emph{flasque} si chacun des $F_n$ est un $A$--Module (ou, ce qui revient au même, un faisceau abélien) flasque.
}
\vskip .3cm
{
Proposition {\bf 4.2.4}. --- \it 
\begin{enumerate}
    \item[(i)] Soit $X$ un topos. Tout $A$-faisceau sur $X$ se plonge dans un $A$-faisceau flasque. 
    \item[(ii)] Soient $X$ et $Y$ deux topos et $f: X \to Y$ un morphisme quasicompact. Pour tout $A$-faisceau flasque $F$ sur $X$, le $A$-faisceau $f_*(F)$ est flasque et on a 
    $$
    \Rd^p f_* (F) = 0 \quad (p \geq 1),~\text{dans}~A-\fsc(Y).
    $$
    En particulier, pour tout entier $p \geq 1$, le foncteur $\Rd^p f_*$ est effa\c{c}able.
\end{enumerate}
}
\vskip .3cm
{\bf Preuve} : Lorsque $X$ admet suffisamment de points, le caractère fonctoriel de la ``résolution de Godement'' (SGA4 XII 3.4) permet de la prolonger aux systèmes projectifs, ce qui montre (i) dans ce cas.

Dans le cas général, on laisse au lecteur le soin de faire la construction pas à pas. D'ailleurs, nous verrons plus loin (6.6.3) que la catégorie $\mathcal{E}(X, J)$ possède suffisamment d'injectifs et que ceux-ci sont flasques, ce qui prouvera également le résultat annoncé. Quant à l'assertion (ii), elle est immédiate.
\vskip .3cm
{\bf 4.3}. Soient $X$ et $Y$ deux topos et $f: X \to Y$ un morphisme quasicompact. Soient $F$ un $A$-faisceau sur $Y$ et $G$ un $A$-faisceau sur $X$.

On définit de fa\c{c}on évidente, composant par composant, des ``morphismes d'adjonction''
$$
a_F: F \to f_* f^*(F)
\leqno{(4.3.1)}
$$
$$
b_G: f^* f_*(G) \to G
\leqno{(4.3.2)}
$$
fonctoriels en $F$ et $G$ respectivement. On en déduit des applications fonctorielles (Sém. CARTAN 11 Exp. 7 par 3)
$$
\varphi: \Hom(f^* F, G) \to \Hom(F, f_*(G))
\leqno{(4.3.3)}
$$
$$
\psi: \Hom(F, f_*(G)) \to \Hom(f^* F, G).
\leqno{(4.3.4)}
$$
\vskip .3cm
{
Proposition {\bf 4.3.5}. --- \it Les applications $\varphi$ et $\psi$ précédentes sont des bijections inverses l'une de l'autre. En particulier, le foncteur
$$
f^*: A-\fsc(Y) \to A-\fsc(X)
$$
est adjoint à gauche du foncteur
$$
f_*: A-\fsc(X) \to A-\fsc(Y).
$$
}
\vskip .3cm
{\bf Preuve} : Il suffit (cf. loc. cit.) de montrer que les composés
$$
f_* (G) \xlongrightarrow{a f_* (F)} f_* f^* f_* (G) \xlongrightarrow{f_*(b_F)} f_* (G)
$$
et
$$
f^*(F) \xlongrightarrow{f^*(a_F)} f^* f_* f^* (F) \xlongrightarrow{b_{f^*(F)}} f^* (F)
$$
sont respectivement l'identité de $f_*(G)$ et celle de $f^*(F)$. Or cela est vrai au stade des composants, d'où l'assertion.
\vskip .3cm
{\bf 4.4}. Soit $X$ un topos dont l'\emph{objet final est quasicompact}, de sorte que (SGA4 VI 3.2), le morphisme canonique
$$
p: X \to \pt
$$
est \emph{quasicompact}. On définit de fa\c{c}on évidente un foncteur cohomologique
$$
(\overline{\mathrm{H}}^i (X, .))_{i \in \mathbf{Z}}: \mathscr{E}(X, J) \to \mathscr{E}(\pt, J) \quad (\to \Pro(A-\mod)),
$$
en posant pour tout $A$-faisceau $F = (F_n)_{n \in \mathbf{N}}$
$$
\overline{\mathrm{H}}^i (X, F) = (\overline{\mathrm{H}}^i(X, F_n))_{n \in \mathbf{N}}
\leqno{(4.4.1)}
$$
avec les morphismes de transition évidents.

Le même raisonnement qu'en (4.2) montre qu'il définit par passage au quotient un nouveau foncteur cohomologique, noté sans inconvénient de la même manière,
$$
(\overline{\mathrm{H}}^i (X, .))_{i \in \mathbf{Z}}: A-\fsc(X) \to A-\fsc(\pt) \hookrightarrow \Pro(A-\mod).
\leqno{(4.4.2)}
$$
Comme précédemment, $\overline{\mathrm{H}}^i = 0$ pour $i < 0$, et on pose $\overline{\mathrm{H}}^0 (X, .) = \overline{\Gamma}(X, .)$.

Identifiant de la fa\c{c}on habituelle les foncteurs $\Gamma$ et $p_*$ pour les $A_X$--Modules, on obtient une identification canonique entre les foncteurs $\overline{\mathrm{H}}^i (X, .)$ et $\Rd^i p_* (.)$, de sorte que les énoncés précédents peuvent être considérés comme une redite de (4.2). 
\vskip .3cm
{\bf 4.4.3}. Soient $X$ et $Y$ deux topos dont l'objet final est quasicompact et $f: X \to Y$ un morphisme. Si $F$ est un $A$-faisceau sur $Y$, l'image réciproque en cohomologie définit de fa\c{c}on évidete un morphisme de foncteurs cohomologiques
$$
f^*: \overline{\mathrm{H}}^p (Y, F) \to \overline{\mathrm{H}}^p (X, f^*(F))
$$
avec les propriétés habituelles (isomorphisme canonique pour le composé, avec condition de cocycles).
\vskip .3cm
{\bf 4.5}. Soient $X$ un topos et $f: T \to T'$ un morphisme quasicompact (SGA4 VI 1.7) de $X$. Le foncteur exact (SGA4 III 6.8)
$$
f_!: A-\Mod_T \to A-\Mod_{T'}
$$
définit de fa\c{c}on claire un foncteur exact
$$
f_!: \mathscr{E}(T, J) \to \mathscr{E}(T', J)
\leqno{(4.5.1)}
$$
en posant pour tout $A$-faisceau $F = (F_n)_{n \in \mathbf{N}}$ sur $\mathbf{T}$
$$
f_!(F) = (f_!(F_n))_{n \in \mathbf{N}}.
\leqno{(4.5.2)}
$$
Comme $f$ est quasicompact, le foncteu (4.5.1) transforme $A$-faisceau négligeable en $A$-faisceau négligeable, et définit par suite par passage au quotient un foncteur exact, noté de même,
$$
f_!: A-\fsc(T) \to A-\fsc(T').
\leqno{(4.5.3)}
$$
\vskip .3cm
{\bf 4.5.4}. Si $g: T' \to T''$ est un autre morphisme quasicompact de $X$, on définit, composant par composant sur les systèmes projectifs, un isomorphisme
$$
(g f)_! \isomlong g_! f_!
$$
vérifiant la condition de cocycles habituelle.
\vskip .3cm
{
Proposition {\bf 4.5.5}. --- \it Soient $X$ un topos et $f: T \to T'$ un morphisme \emph{quasicompact} de $X$. Le foncteur
$$
f_!: A-\fsc(T) \to A-\fsc(T')
$$
est adjoint à gauche du foncteur
$$
f^*: A-\fsc(T') \to A-\fsc(T).
$$
}
\vskip .3cm
{\bf Preuve} : On se ramène comme dans la preuve de (4.3.5) à l'assertion analogue pour les $A$--Modules.
\vskip .3cm
{\bf 4.6}. Soient $X$ un topos, $U$ un ouvert de $X$ et $Y$ le fermé complémentaire de $U$ (SGA4 IV 3.3). On note
$$
j: Y \to X
$$
le morphisme de topos canonique, et on rappelle que $j$ est quasicompact. Sur le modèle de (4.2), le foncteur cohomologique (\quad)
$$
(\Rd^p j^!)_{p \in \mathbf{Z}}: A-\Mod_X \to A-\Mod_Y
$$
permet de définir un foncteur cohomologique, noté de même, 
$$
(\Rd^p j^!)_{p \in \mathbf{Z}}: A-\fsc(X) \to A-\fsc(Y).
\leqno{(4.6.1)}
$$
On a $\Rd^i j^! = 0$ pour $i < 0$, et on pose $j^! = \Rd^0 j^!$. Le foncteur $j^!$ est \emph{exact à guache}.
\vskip .3cm
{\bf 4.6.2}. Si $k: Z \to Y$ est une autre immersion fermée de topos, on a un isomorphisme canonique
$$
(jk)^! \isomlong (k^!)(j^!),
$$
vérifiant la condition de cocycles habituelle.
\vskip .3cm
{
Proposition {\bf 4.6.3}. --- \it Le foncteur
$$
j_*: A-\fsc(Y) \to A-\fsc(X)
$$
est adjoint à guache du foncteur
$$
j^!: A-\fsc(X) \to A-\fsc(Y).
$$
}
\vskip .3cm
{\bf Preuve} : Analogue à celle de (4.3.5), compte tenu de (SGA4 IV 3.6).
\vskip .3cm
{
Proposition {\bf 4.6.4}. --- \it On suppose que le morphisme canonique $i: U \to X$ est \emph{quasicompact}. Alors on a pour tout $A$-faisceau $F$ sur $X$ des suites exactes de $\mathcal{E}(X, J)$, donc aussi de $A-\fsc(X)$, fonctorielles en $F$,
$$
0 \to i_! i^*(F) \to F \to j_* j^*(F) \to 0
\leqno{(i)}
$$
$$
0 \to j_* j^! (F) \to F \to i_* i^*(F),
\leqno{(ii)}
$$
dans lesquelles les flèches non évidentes désignent les morphismes d'adjonction. 
}
\vskip .3cm
{\bf Preuve} : Résulte aussitôt de (SGA4 IV 3.7) appliqué aux composants de $F$.
\vskip .3cm
{\bf 4.6.5}. Signalons enfin que toutes les opérations que nous venons de définir transforment évidemment $A$-faisceau de type constant en $A$-faisceau de type constant, et que les foncteurs image réciproque et prolongement par zéro (4.5.3), étant exacts, transforment $A$-faisceau de type $J$-adique en $A$-faisceau de type $J$-adique. Nous verrons dans le chapitre II, moyennant des conditions de finitude convenables, d'autres propriétés de stabilité pour ces notions.
