%%%%%%%%%%%%%%%%%%%%%%%%%%%%%%%%%%%%
\subsection*{8. Changement d'anneau.}
\addcontentsline{toc}{subsection}{8. Changement d'anneau}

\vskip .3cm
{\bf 8.1}. Soient $X$ un topos, $A$ et $B$ deux anneaux commutatifs unifères, $J$ et $K$ deux idéaux de $A$ et $B$ respectivement et
$$
u: A \to B
$$
un morphisme d'anneaux unifères, vérifiant $u(J) \subset K$. Comme il n'y aura pas d'ambiguïté, on appellera $A$-faisceaux les $(A, J)$-faisceaux et $B$-faisceaux les $(B, K)$-faisceaux (1.2).

Il est clair que tout $B$-faisceau $F$ est canoniquement muni d'une structure de $A$-faisceau d'où un foncteur exact
$$
\mathscr{E}(X, K) \to \mathscr{E}(X, J),
$$
qui fournit par passage au quotient un foncteur \emph{exact et conservatif}
$$
u_*: B-\fsc(X) \to A-\fsc(X),
\leqno{(8.1.1)}
$$
d'où un foncteur exact noté de même sur les catégories dérivées
$$
u_*: \D(X, B) \to \D(X, A).
\leqno{(8.1.2)}
$$
Étant donné un $A$-faisceau $F$, le $A$-faisceau $F \otimes_A u_*(B)$ est canoniquement muni d'une structure de $B$-faisceau grâce aux structures de $B$--Modules des composants du deuxième facteur, d'où un foncteur \emph{exact à droite}
$$
u^*: A-\fsc(X) \to B-\fsc(X).
\leqno{(8.1.3)}
$$
Utilisant des résolutions plates dans la catégorie des $A$-faisceaux, on en déduit un foncteur exact
$$
\bLd u^*: \D^-(X, A) \to \D^-(X, B).
\leqno{(8.1.4)}
$$
De plus, lorsque $A$ est un anneau local régulier et $J$ son idéal maximal, il résulte de (5.16) que le foncteur (8.1.4) se prolonge, toujours en utilisant des résolutions plates, en un foncteur noté de même
$$
\bLd u^*: \D(X, A) \to \D(X, B),
\leqno{(8.1.5)}
$$
qui envoie $\D^+(X, A)$ dans $\D^+(X, B)$.
\vskip .3cm
{
Proposition {\bf 8.1.6}. --- \it Soient $E \in \D(X, A)$ et $F \in \D(X, B)$.
\begin{itemize}
    \item[(i)] Si $E \in \D^-(X, A)$ (resp. $A$ est local régulier et $J$ est son idéal maximal), on a un isomorphisme fonctoriel de $A$-modules
    $$
    \Hom_B(\bLd u^* (E), F) \isomlong \Hom_A(E, u_*(F)).
    $$
    \item[(ii)] Si $E \in \D^-(X, A)$ et $F \in \D^+(X, B)$, on a un isomorphisme fonctoriel
    $$
    u_* \bRd \cHom_B (\bLd u^*(E), F) \isomlong \bRd \cHom_A(E, u_*(F)).
    $$
    \item[(iii)] Si le topos $X$ et l'anneau $A$ sont noethériens, et si $E \in \D^-(X, A)$ et $F \in \D^+(X, B)$, on a des isomorphismes fonctoriels
    $$
    u_* \bRd \overline{\Hom}_B(\bLd u^*(E), F) \isomlong \bRd \overline{\Hom}_A(E, u_*(F))
    $$
    et
    $$
    u_* \bRd \overline{\Hom}_B(\bLd u^*(E), F) \isomlong \bRd \Hom_A(E, u_*(F))
    $$
    dans $\D(A-\fsc(\pt))$ et $\D(A-\mod)$ respectivement, en convenant de noter encore $u: \D(B-\mod) \to \D(A-\mod)$ le foncteur restriction des scalaires.
\end{itemize}
}
\vskip .3cm
{\bf Preuve} : Montrons (i). Nous allons pour cela définir tout d'abord un morphisme d'``adjonction''
$$
E \to u_* \bLd u^* (E).
\leqno{(8.1.7)}
$$
Pour ce faire, on peut, dans chacun des cas envisagés, supposer $E$ plat, et on prend alors le morphisme ``extension des scalaires''
$$
E \to E \otimes_A u_* (B)
$$
déduit du morphisme évident de $A$-faisceaux $A \to u_* (B)$. Le morphisme (8.1.7) permet de définir de fa\c{c}on évidente un morphisme de bifoncteurs cohomologiques
$$
\Hom_B(\Ld u^* (E), F) \to \Hom_A(E, u^* (F)),
$$
et il s'agit de voir que c'est un isomorphisme. Quitte à découper $F$ en parties positive et négative, on est ramené à le voir lorsque $F \in \D^+(X, B)$ ou $F \in \D^-(X, B)$. Dans l'hypothèse non respée, le cas où $F \in \D^+(X, B)$ se ramène, au moyen du ``way-out functor lemma'', au cas où $F$ est borné, de sorte qu'on peut pour prouver l'assertion supposer que $F \in \D^-(X, B)$. Alors, on définit comme suit un morphisme d'``adjonction''
$$
\bLd u^* u_* F \to F.
$$
On choisit une résolution $A$-plate $P \to u_*(F)$, et la structure de $B$-faisceaux sur les composants de $u_*(F)$ permet d'en déduire de fa\c{c}on évidente un morphisme de complexes $u^*(P) \to F$ qui répond à la question. Dans l'hypothèse respée, le morphisme (8.1.8) se définit de même sans hypothèse de degrés sur $F$. Enfin, les morphismes composés canoniques déduits de (8.1.7) et (8.1.8) sont les identités, d'où (i). Pour montrer l'assertion (ii), on peut supposer $E$ quasilibre et $F$ flasque, ce qui implique que $u^*(E)$ et $u_*(F)$ sont respectivement quasilibre et flasque. Le lemme suivant permet alors de définir un isomorphisme de complexes
$$
u_* \cHom^\bullet_B(u^*(E), F) \isomlong \cHom^\bullet_A(E, u_*(F)),
$$
qui répond à la question.
\vskip .3cm
{
Lemme {\bf 8.1.9}. --- \it Étant donnés un $A$-faisceau $E = (E_n)_{n \in \mathbf{N}}$ et un $B$-faisceau $F = (F_n)_{n \in \mathbf{N}}$, il existe un isomorphisme fonctoriel
$$
u_* \cHom_B(u^*(E), F) \isomlong \cHom_A(E, u_*(F)).
$$
}
\vskip .3cm
On va le définir sur les composants. Soit $n$ un entier $\geq 0$; pour tout entier $m \geq n$, on a un isomorphisme évident de $A_m$--Modules
$$
\cHom_{B_m}(E_m \otimes_{A_m}B_m, F_n) \isomlong \cHom_{A_m}(E_m, F_n),
$$
et on obtient l'isomorphisme désiré sur les composants d'ordre $n$ en passant à la limite inductive suivant $m$.

Compte tenu de (7.4.18), l'assertion (iii) résulte de (ii) et des isomorphismes 
\[\begin{tikzcd}
	{\bRd \overline{\Gamma} \circ u_*} && {u_* \circ \bRd \overline{\Gamma}} \\
	{\bRd \Gamma \circ u_*} && {u_* \circ \bRd \Gamma,}
	\arrow["\sim", from=1-1, to=1-3]
	\arrow["\sim", from=2-1, to=2-3]
\end{tikzcd}\leqno{(8.1.10)}\]
qui proviennent immédiatemment du fait que la propriété pour un $A$--Module d'être flasque ne dépend pas de l'Anneau de base.
\vskip .3cm
{
Proposition {\bf 8.1.11}. --- \it
\begin{itemize}
    \item[(i)] Si $L$ est un $A$-faisceau plat, le $B$-faisceau $u^*(L)$ est plat. Si $E \in D^-(X, A)_{\torf}$, alors $\bLd u^*(E) \in \D^-(X, B)_{\torf}$.
    \item[(ii)] Soient $E$ et $F \in \D(X, A)$. On a un isomorphisme fonctoriel
    $$
    \bLd u^* (E \underline{\otimes}_A F) \isomlong \bLd u^* (E) \underline{\otimes}_B \bLd u^*(F)
    $$
    dans chacun des cas suivants:
    \begin{itemize}
        \item $E$ et $F \in \D^-(X, A)$.
        \item $A$ et $B$ sont locaux réguliers, $J$ et $K$ sont leurs idéaux maximaux, et $E$ et $F \in \D^+(X, A)$, ou $E \in \D^b(X, A)$.
    \end{itemize}
    \item[(iii)] Soient $E \in \D(X, A)$ et $F \in \D(X, B)$. On a un isomorphisme fonctoriel
    $$
    u_* (\bLd u^* (E) \underline{\otimes}_B F) \isomlong u_*(F) \underline{\otimes}_A E,
    $$
    appelé \emph{formule de projection}, dans chacun des cas suivants :
    \begin{itemize}
        \item $E \in \D^-(X, A)$ et $F \in \D^-(X, B)$. 
        \item $E \in \D^-(X, A)_{\torf}$.
        \item $A$ et $B$ son locaux réguliers, $J$ et $K$ sont leurs idéaux maximaux, $E \in \D^+(X, A)$ et $F \in \D^+(X, B)$.
    \end{itemize}
    \item[(iv)] Soient $E \in \D^-(X, A)$ et $F \in \D^+(X, A)$. On a un morphisme canonique fonctoriel
    $$
    \bLd u^* \bLd \cHom_A (E, F) \to \bRd \cHom_B (\bLd u^* (E), \bLd u^*(F))
    $$
    lorsque $A$ est un anneau local régulier et $J$ son idéal maximal. Ce morphisme est un \emph{isomorphisme} lorsque de plus $B$ est une $A$-algèbre finis et $K = JB$.
    \item[(v)] Soient $E \in \D^-(X, A)$ et $F \in \D^+(X, A)$. Lorsque l'objet final de $X$ est quasicompact, et que l'anneau $A$ est régulier et $J$ est son idéal maximal, on a des morphismes canoniques
    $$
    \bRd \overline{\Hom}_A(E, F) \to u_* \bRd \overline{\Hom}_B(\bLd u^* (E), \bLd u^*(F))
    $$
    $$
    \bRd \Hom_A(E, F) \to u_* \bRd \Hom_B(\bLd u^* (E), \bLd u^*(F)).
    $$
\end{itemize}
}
\vskip .3cm
{\bf Preuve} : Étant donnés un $A$-faisceau $L$ et un $B$-faisceau $M$, on définit composant par composant un isomorphisme ``de projection''
$$
u_* (u^*(L) \otimes_B M) \isomlong L \otimes_A u_* (M)
\leqno{(8.1.12)}
$$
qui nous servira dans la preuve de (iii). Lorsque $L$ est plat, on en déduit que $u^*(L)$ est plat, grâce à l'exactitude et à la conservativité du foncteur $u_*$ ; l'autre partie de (i) en résulte aussitôt. Pour voir (ii), on peut dans chacun des cas de l'énoncé prendre $E$ et $F$ plats, et alors on exhibe de fa\c{c}on évidente un isomorphisme de complexes qui répond à la question. De même pour (iii), compte tenu de l'isomorphisme (8.1.12). Montrons (iv). Par (8.1.6.(i)), il s'agit de définir un morphisme
$$
\bRd \cHom_A(E, F) \to u_* \bRd \cHom_B (\bLd u^* (E), \bLd u^*(F)),
$$
soit encore, par (8.1.6.(ii)), un morphisme
$$
\bRd \cHom_A(E, F) \to \bRd \cHom_A (E, u_* \bLd u^* (F)).
$$
On prend celui déduit de fa\c{c}on évidente du morphisme d'adjonction (8.1.7) : $F \to u_* \bLd u^* (F)$. Notant $q$ le morphisme de l'énoncé, nous allons donner une description directe de $u_* (q)$. Il s'agit, compte tenu de (8.1.16.(iii)), d'un morphisme 
$$
B \underline{\otimes}_A \bRd \cHom_A (E, F) \to \bRd \cHom_A (E, B \underline{\otimes}_A F),
\leqno{(8.1.13)}
$$
qui n'est autre, comme on s'en assure aisément, que (7.6.9.2). Supposons maintenant que $B$ soit une $A$-algèbre finie et que $K = JB$. Pour voir que dans ce cas le morphisme $q$ est un isomorphisme, il suffit, d'après la conservativité du foncteur $u_*$, de montrer que (8.1.13) en est un. Quitte à remplacer $B$ par une résolution finie par des $A$-modules libres de type fini, on est ramené au cas où $B = A$, et alors l'assertion est évidente. Prouvons enfin (v). Nous allons le faire pour les $\bRd \overline{\Hom}$, la construction dans l'autre cas étant analogue. Appliquant le foncteur $u_* \circ \bRd \overline{\Gamma}$ au morphisme $q$, on obtient un morphisme
$$
u_* \bRd \overline{\Gamma} \bLd u^* \bRd \cHom_A (E, F) \to \bRd \cHom_B (\bLd u^* (E), \bLd u^* (F)).
\leqno{(8.1.14)}
$$
Par ailleurs, le morphisme d'adjonction $\id \xlongrightarrow{a} u_* \bLd u^*$ définit, grâce à(8.1.10), un morphisme
$$
\bRd \overline{\Gamma}(a): \bRd \overline{\Gamma} \bRd\cHom_A (E, F) \to u_* \bRd \overline{\Gamma} \bLd u^* \bRd \cHom_A(E, F)
\leqno{(8.1.15)}
$$
soit, d'après (7.4.18.(i)),
$$
\bRd \overline{\Hom}_A(E, F) \to u_* \bRd \overline{\Gamma} \bRd \cHom_A (E, F).
\leqno{(8.1.15)bis}
$$
Le morphisme annoncé est le composé de (8.1.14) et (8.1.15)bis.
\vskip .3cm
{
Proposition {\bf 8.1.16}. --- \it  
\begin{itemize}
    \item[(i)] Soit $f: X \to Y$ un morphisme de topos. Étant donné $E \in \D(Y, A)$, on a un \emph{isomorphisme} fonctoriel
    $$
    \bLd u^* f^* (E) \to f^* \bLd u^* (E)
    $$
    lorsque $E \in \D^-(Y, A)$ ou $A$ est local régulier et $J$ est son idéal maximal. 
    \item[(ii)] Soient $X$ un topos, $T$ et $T'$ deux objets de $X$ et $i: T \to T'$ un morphisme quasicompact. Étant donné $E \in \D^(T, A)$, on a un \emph{isomorphisme} fonctoriel
    $$
    \bRd i_! \bLd u^* (E) \isomlong \bLd u^* \bRd i_! (E).
    $$
    lorsque $E \in \D^-(T, A)$, ou $A$ est local régulier et $J$ est son idéal maximal.
    \item[(iii)] Soient $X$ un topos, $U$ un ouvert de $X$, $Y$ le topos fermé complémentaire et $j: Y \to X$ le morphisme de topos canonique. Lorsque $A$ est régulier et $J$ est son idéal maximal, on a pour tout $E \in \D^+(X, A)$ un morphisme fonctoriel
    $$
    \bLd u^* \bRd j^! (E) \to \bRd j^! \bLd u^* (E).
    $$
    Ce morphisme est un \emph{isomorphisme} lorsque de plus $B$ est une $A$-algèbre finie et $K = JB$.
    \item[(iv)] Sous les hypothèses préliminaires de (iii), étant donné $E \in \D(Y, A)$, on a un \emph{isomorphisme} fonctoriel
    $$
    \bLd u^* \bRd j_* (E) \isomlong \bRd j_* \bLd u^* (E)
    $$
    lorsque $E \in \D^-(Y, A)$, ou $A$ est local régulier et $J$ est son idéal maximal.
\end{itemize}
}
\vskip .3cm
{\bf Preuve} : Montrons (i). Lorsque $E \in \D^-(Y, A)$, on peut le supposer quasilibre, de sorte que $f^*(E)$ l'est également (5.17.2). Comme $f^*(B) \isommap B$, l'isomorphisme évident de complexes
$$
B \otimes_A f^*(E) \to f^*(B \otimes_A E),
$$
défini composant par composant, répond à la question. Dans la deuxième hypothèse, on peut supposer $E$ plat, et on concluera comme précédemment si on prouve que $B \otimes_A f^* (E) \isom B \underline{\otimes}_a f^* (E)$. Comme le foncteur $\bLd u^*$ est de dimension cohomologique finie à gauche, il suffit pour cela (CD début page 43), de montrer que pour tout $A$-faisceau plat $M$ sur $Y$, on a $\cTor^A_i(B, f^*(M)) = 0$ $(i \geq 1)$, ce qui est immédiat à partir de (5.17.1). Pour voir (ii), on peut dans chacun des cas considérés supposer $E$ plat, de sorte que $i_!(E)$ l'est aussi (5.18.5.(i)). 

Alors, on construit, composant par composant à partir de (5.18.1), un isomorphisme de complexes
$$
i_!(E \otimes_A i^*(B)) \isomlong i_!(E) \otimes_A B
$$
qui répond à la question. L'assertion (iv) se montre de même à partir de (5.19). Prouvons (iii). D'après (7.7.13), il s'agit de définir un morphisme
$$
\bLd u^* j^* \bRd \cHom_A (j_* A, E) \to j^* \bRd \cHom_B (j_* B, \bLd u^*(E)),
\leqno{(8.1.17)}
$$
ou encore, compte tenu de (i), un morphisme
$$
j^* \bLd u^* \bRd \cHom_A (j_* A, E) \to j^* \bRd \cHom_B (\bLd u^* j_* A, \bLd u^* (E)).
\leqno{(8.1.18)}
$$
On prend celui déduit de (8.1.11.(iv)) par application du foncteur $j^*$. L'assertion d'isomorphie résulte également de (loc. cit.). 
\vskip .3cm
{
Définition {\bf 8.1.19}. --- \it Étant donnés deux idéotopes $(X, A, J)$ et $(Y, B, K)$, on appelle \emph{morphisme d'idéotopes} $(X, A, J) \to (Y, B, K)$ un couple $(f, u)$ formé d'un morphisme de topos $f: X \to Y$ et d'un morphisme d'anneaux unifères $u: B \to A$ vérifiant
$$
u(K) \subset J.
$$
}
\vskip .3cm
On vérifie immédiatement qu'on définit ainsi une catégorie, appelée \emph{catégorie des idéotopes}. Étant donné un morphisme d'idéotopes
$$
(f, u) = \varphi: (X, A, J) \to (Y, B, K),
$$
on note $\varphi^*$ et on appelle \emph{image réciproque} par $\varphi$ le foncteur exact 
$$
\varphi^* = u^* \circ f^* : \B-\fsc(Y) \to A-\fsc(X).
$$
De même, on note $\bLd \varphi^*$ le foncteur exact
$$
\bLd u^* \circ f^* : \D^-(Y, B) \to \D^-(X, A),
$$
qui se prolonge lorsque $B$ est régulier et $K$ est son idéal maximal, en un foncteur exact noté de même de $\D(Y, B)$ dans $\D(X, A)$.

De fa\c{c}on analogue, on définit les foncteurs \emph{images directes} par 
$$
\varphi_* = u_* \circ f_* : A-\fsc(X) \to B-\fsc(Y)
$$
et 
$$
\bRd \varphi_* = u_* \circ \bRd f_* : \D^+(X, A) \to \D^+(Y, B).
$$
On déduit sans peine de (7.7) et (8.1) des formalismes analogues pour les morphismes d'idéotopes, qu'on laisse au lecteur le soin d'expliciter, car nous n'en aurons pas besoin dans la suite.
 
\vskip .3cm
{\bf 8.2}. Nous allons maintenant expliciter dans quelques cas particuliers les résultats du numéro précédent.
\vskip .3cm
{
Proposition {\bf 8.2.1}. --- \it Soit $(X, A, J)$ un idéotope. Pour tout entier $n > 0$, les foncteurs
$$
(\id_A)_*: (A, J)-\fsc(X) \to (A, J^n)-\fsc(X)
$$
et
$$
(\id_A)^*: (A, J^n)-\fsc(X) \to (A, J)-\fsc(X)
$$
sont des équivalences quasi-inverses l'une de l'autre. En particulier, lorsque $A$ est noethérien, la catégorie $A-\fsc(X)$ ne dépend pas (à équivalence près) de l'idéal $J$, mais seulement de la topologie qu'il définit (ce qui justifie a posteriori la notation $A-\fsc(X)$ au lieu de $(A, J)-\fsc(X)$, étant alors sous-entendu que la lettre $A$ désigne un anneau topologique).
}
\vskip .3cm
{\bf Preuve} : Il est immédiat que les morphismes d'adjonction $\id \to u_* u^*$ et $u^*u_* \to \id$ sont des isomorphismes (on pose $u = \id$).

A partir de maintenant, on suppose donné un idéotope $(X, A, J)$, un entier $n \geq 0$, et on note $u: A \to A_n = A/J^{n+1}$ le morphisme d'anneaux canonique. L'anneau $A_n$ sera toujours supposé muni de l'idéal $J/J^{n+1}$. 
\vskip .3cm
{
Proposition {\bf 8.2.2}. --- \it Sous les conditions précédentes, le foncteur
$$
\bLd u^*: \D^-(X, A) \to \D^-(X, A_n)
$$
est \emph{conservatif}.
}
\vskip .3cm
{\bf Preuve} : Soit $p$ un morphisme de $\D^-(X, A)$ tel que $\bLd u^*(p)$ soit un isomorphisme, et montrons que $p$ est un isomorphisme. Comme $\bLd u^*$ transforme triangle exact en tringle exact, on est ramené, par considération du mapping-cylinder de $p$, à montrer que si $E$ est un objet de $\D^-(X, A)$ tel que $\bLd u^* (E)$ soit acyclique, alors $E$ est acyclique. Nous allons pour cela raisonner par l'absurde et supposer qu'il existe un plus grand entier $i$ tel que $\mathrm{H}^i(E) \neq 0$ ; quitte à tronquer $E$ et à translater les degrés, on peut d'ailleurs supposer que $E$ est à degrés $\leq 0$ et que $i = 0$, puis même que $E$ est plat. Alors il est clair que $\mathrm{H}^0(\bLd u^*(E)) = \mathrm{H}^0(E)/J^{n+1}\mathrm{H}^0(E)$, d'où $\mathrm{H}^0(E) = J^{n+1}\mathrm{H}^0(E)$ par hypothèse et $\mathrm{H}^0(E) = 0$ par le ``lemme de Nakayama'' (5.12). D'où la contradiction annoncée.

Nous allons maintenant comparer les $A_n$--Modules et les $A_n$-faisceaux. Tout d'abord, on sait (3.6) que le foncteur canonique (3.5.1)
$$
A_n-\Mod_X \to A_n-\fsc(X)
\leqno{(8.2.3)}
$$
est exact, pleinement fidèle, et que son image essentielle est une sous-catégorie exacte de $A_n-\fsc(X)$. Pour des raisons de commodité, nous noterons ici
$$
A_n-\fsc_0(X)
$$
cette sous-catégorie exacte de $A-\fsc(X)$. On en déduit que la sous-catégorie
$$
\D_0(X, A_n)
$$
de $\D(X, A_n)$ engendrée par les complexes dont la cohomologie appartient à $A_n-\fsc_0(X)$ est une sous-catégorie triangulée. Même remarque pour les catégories analogues $\D^*_0(X, A_n)$, avec des notations évidentes. Le foncteur (8.2.3) définit des foncteurs exacts
$$
\omega^*: \D^*(A_n-\Mod_X) \to \D^*_0(X, A_n).
\leqno{(8.2.4)}
$$
\vskip .3cm
{
Proposition {\bf 8.2.5}. --- \it  
\begin{itemize}
    \item[(i)] Le foncteur (8.2.4) induit un foncteur
    $$
    \D^-(A_n-\Mod_X)_{\torf} \to \D^-(X, A_n)_{\torf}.
    $$
    \item[(ii)] Soient $E$ et $F \in \D(A_n-\Mod_X)$. On a un isomorphisme fonctoriel
    $$
    \omega(E) \underline{\otimes} \omega(F) \isomlong \omega(E \underline{\otimes}F)
    $$
    lorsque $E$ et $F \in \D^-(A_n-\Mod_X)$, où $E \in \D^-(A_n-\Mod_X)_{\torf}$.
    \item[(iii)] Si $E \in \D^-(A_n-\Mod_X)$ et $F \in \D^+(A_n-\Mod_X)$, on a un isomorphisme fonctoriel
    $$
    \omega \bRd \cHom_{A_n}(E, F) \isomlong \bRd \cHom_{A_n}(\omega(E), \omega(F)).
    $$
\end{itemize}
}
\vskip .3cm
{\bf Preuve} : L'assertion (i) résulte de ce que un $A_n$--Module plat définit un $A_n$-faisceau fortement plat. Profitons d'ailleurs de l'occassion pour remarquer qu'un $A_n$--Module libre (sur un faisceau d'ensembles) définit un $A_n$-faisceau quasilibre. L'assertion (ii) se voit sans peine, en prenant $E$ plat. Pour (iii), on peut supposer $E$ quasilibre (i.e. à composants des $A_n$--Modules libres) et $F$ flasque. Alors $\omega(E)$ est quasilibre et $\omega(F)$ a des composants qui sont isomorphes à des $A_n$-faisceaux flasques, d'où aussitôt l'assertion.
\vskip .3cm
{
Proposition {\bf 8.2.6}. --- \it On suppose que l'objet final de $X$ est quasicompact. Alors le foncteur (8.2.4)
$$
\omega^+_0: \D^+(A_n-\mod_X) \to \D^+_0(X, A_n)
$$
est une \emph{équivalence de catégories}.
}
\vskip .3cm
{\bf Preuve} : Résulte immédiatement de (7.1.2).
\vskip .3cm
{\bf Notations 8.2.7}. Dans la suite, on notera
$$
\alpha: A_n-\Mod_X \to A-\fsc(X)
$$
le foncteur canonique, composé de $u_*$ et de (8.2.3). On posera
$$
\alpha_* = u_* \circ \omega: \D(A_n-\Mod_X) \to \D(X, A).
$$
Avant de poursuivre, introduisons une nouvelle notion. Notant pour tout entier $p \geq 0$ par $u_p: A \to A_p$ le morphisme d'anneaux canonique, la sous-catégorie pleine de $\D^-(X, A)$ (resp. $\D^-(X, A)_{\torf}$) engendrée par les objets $E$ vérifiant
$$
\bLd u^*_p(E) \in \D_0(X, A_p) \quad (p \in \mathbf{N})
$$
est une sous-catégorie triangulée. On la note
$$
\D^-_0(X, A) \quad (\text{resp.}~\D^-_0(X, A)_{\torf}).
$$
Si $A$ est local régulier et $J$ est son idéal maximal, on définit de même la catégorie triangulée $\D_0(X, A)$. La cohérence de cette notation est donnée par la proposition suivante :
\vskip .3cm
{
Proposition {\bf 8.2.8}. --- \it Soit $E \in \D^-_0(X, A_n)$. Alors pour tout entier $p \leq n$, on a, en notant $v_p: A_n \to A_p$ le morphisme d'anneaux canonique,
$$
\bLd v^*_p(E) \in \D^-_0(X, A).
$$
}
\vskip .3cm
{\bf Preuve} : D'après ([H] I 7.3.), on peut supposer que $E$ est ``réduit au degré 0''. Notant de même le $A_n$--Module correspondant et $P$ une résolution de $E$ par des $A_n$--Modules plats, on a alors
$$
\bLd v^*_p (E) \quad \text{``=''} \quad  A_p \otimes_{A_n} P,
$$
d'où l'assertion.

Remarquons que l'intérêt de la catégorie $\D^-_0(X, A)$ vient de ce qu'elle contient, lorsque $X$ est localement noethérien, la catégorie triangulée correspondante formée des complexes à cohomologie des $A$-faisceaux \emph{constructibles}.

Supposons maintenant que l'objet final de $X$ soit quasicompact. Choisissant un foncteur quasi-inverse $\omega^{-1}$ de $\omega^+_0$ (8.2.6), nous noterons
$$
\bLd \alpha^* = \omega^{-1} \circ \bLd u^*: \D^b_0(X, A)_{\torf} \to \D^b(A_n-\Mod_X)_{\torf}.
\leqno{(8.2.9)}
$$
Le ``$\torf$'' en indice dans le second membre provient de (8.1.11.(i)) et de (8.2.5.(i)). Lorsque de plus l'anneau $A$ est local régulier et $J$ est son idéal maximal, on note de même le foncteur
$$
\bLd \alpha^* = \omega^{-1} \circ \bLd u^*: \D^+_0(X, A) \to \D^+(A_n-\Mod_X).
\leqno{(8.2.9)\text{bis}}
$$
\vskip .3cm
{
Proposition {\bf 8.2.10}. --- \it Soient $E$ et $F \in \D(X, A)$.
\begin{itemize}
    \item[(i)] Si $E$ et $F \in \D^-_0(X, A)$, alors $E \underline{\otimes}_A F \in \D^-_0(X, A)$. 
    \item[(ii)] Si l'anneau $A$ est local régulier et $J$ est son idéal maximal, et si $E$ et $F \in \D^+_0(X, A)$, alors $E \underline{\otimes}_A F \in \D^+_0(X, A)$.
    \item[(iii)] On suppose que l'anneau $A$ est local régulier et que $J$ est son idéal maximal. Alors, si $E \in \D^-_0(X, A)$ et $F \in \D^+_0(x, A)$, on a
    $$
    \bRd \cHom_A(E, F) \in \D^+_0(X, A).
    $$
\end{itemize}
}
\vskip .3cm
{\bf Preuve} : Montrons (i). D'après (8.1.11.(ii)), on est ramené à voir que si $L$ et $M \in \D^-_0(X, A_p)$ pour un entier $p \geq 0$, alors il en est de même pour $L \underline{\otimes}_{A_p} M$. On peut ([H] I 7.3.) supposer que $L$ et $M$ sont réduits au degré 0, et appartiennent donc à $A_p-\fsc_0(X)$. Alors l'assertion résulte de (8.2.5.(ii)). L'assertion (ii) résulte également sans peine de (8.2.6) et (8.2.5. (ii)). Montrons (iii). Grâce à (8.1.11. (iv)), on est ramené à voir l'assertion analogue dans $\D(X, A_p)$ pour tout entier $p \geq 0$. Par dévissage ([H] I 7.3.), on peut supposer que $E$ est réduit au degré 0, et alors l'assertion résulte de (8.2.6) et (8.2.5. (iii)).
\vskip .3cm
{
Proposition {\bf 8.2.11}. --- \it On suppose que l'objet final de $X$ est quasicompact. Si $E \in \D^b_0(X, A)_{\torf}$ et $F \in \D^+(A_n-\Mod_X)$, on a des isomorphismes fonctoriels
$$
\alpha_* \bRd \cHom_{A_n} (\bLd \alpha^*(E), F) \isomlong \bRd \cHom_A (E, \alpha_* F).
\leqno{(i)}
$$
$$
u_* \bRd \cHom_{A_n} (\bLd \alpha^*(E), F) \isomlong \bRd \Hom_A(E, \alpha_* F).
\leqno{(ii)}
$$
Si $E \in \D^b_0(X, A)$ et $F \in \D^+(A_n-\Mod_X)$, ou bien si $A$ est régulier, $J$ est son idéal maximal et $E \in \D^+_0(X, A)$ et $F \in D^+(A_n-\Mod_X)$, on a un isomorphisme fonctoriel
$$
u_* \Hom_{A_n} (\bLd \alpha^*(E), F) \isomlong \Hom_A(E, \alpha_* (F)).
\leqno{(iii)}
$$
}
\vskip .3cm
{\bf Preuve} : Conséquence immédiate des définitions (8.2.7), (8.2.9) et (8.2.9)bis, et de (8.1.6).
\vskip .3cm
{
Proposition {\bf 8.2.12}. --- \it On suppose que l'objet final de $X$ est quasicompact. Soient $E$ et $F \in \D(X, A)$ et $G \in \D(A_n-\Mod_X)$.
\begin{itemize}
    \item[(i)] On a un isomorphisme fonctoriel
    $$
    \bLd \alpha^* (E) \underline{\otimes}_{A_n} \bLd \alpha^*(F) \isomlong \bLd \alpha^*(E \underline{\otimes}_A F)
    $$
    lorsque $E \in \D^b_0(X, A)_{\torf}$ et $F \in \D^b_0(X, A)_{\torf}$, ou bien lorsque $A$ est local régulier, $J$ est son idéal maximal, $E \in \D^b_0(X, A)$ et $F \in \D^+_0(X, A)$.
    \item[(ii)] On suppose que $A$ est local régulier et que $J$ est son idéal maximal. Si $E \in \D^-_0(X, A)$ et $F \in \D^+_0(X, A)$, on a un isomorphisme fonctoriel
    $$
    \bLd \alpha^* \bRd \cHom_A (E, F) \isomlong \bRd \cHom_{A_n} (\bLd \alpha^* (E), \bLd \alpha^*(F)).
    $$
    \item[(iii)] On suppose que $G \in \D^b(A_n-\Mod_X)$ et $F \in \D^b_0(X, A)_{\torf}$, ou bien que $A$ est régulier d'idéal maximal $J$, et que $G \in \D^b(A_n-\Mod_X)_{\torf}$ (resp. $\D^+(A_n-\Mod_X)$) et $F \in \D^+_0(X, A)$ (resp. $\D^b_0(X, A)$). On a alors un isomorphisme fonctoriel
    $$
    \alpha_*(G \underline{\otimes}_{A_n} \bLd \alpha^*(F)) \isomlong \alpha_* (G) \underline{\otimes}_A F.
    $$
\end{itemize}
}
\vskip .3cm
{\bf Preuve} : Résulte formellement de (8.1.11), compte tenu des définitions (8.2.7), (8.2.9)bis.

Signalons enfin quelques propriétés de stabilité par morphismes des catégories $\D_0(X, A)$.
\vskip .3cm
{
Proposition {\bf 8.2.13}. --- \it Soient $A$ un anneau commutatif unifère et $J$ un idéal de $A$.
\begin{itemize}
    \item[(i)] Étant donné un morphisme de topos $f: X \to Y$, le foncteur $f^*$ induit un foncteur
    $$
    \D^-_0(Y, A) \to \D^-_0(X, A).
    $$
    Si de plus $A$ est local régulier et $J$ est son idéal maximal, il induit un foncteur
    $$
    \D_0(Y, A) \to \D_0(X, A).
    $$
    \item[(ii)] Soient $X$ un topos, $T$ et $T'$ deux objets de $X$ et $i: T \to T'$ un morphisme quasicompact. Alors le foncteur $i_!$ induit un foncteur
    $$
    \D^-_0(T, A) \to \D^-_0(T', A)
    $$
    $$
    (\text{resp}.~\D^-_0(T, A)_{\torf} \to \D^-_0(T', A)_{\torf}).
    $$
    Si de plus $A$ est local régulier et $J$ est son idéal maximal, il induit un foncteur
    $$
    \D_0(T, A) \to \D_0(T', A).
    $$
    \item[(iii)] Soient $X$ un topos, $U$ un ouvert de $X, Y$ le topos fermé complémentaire et $j: Y \to X$ le morphisme de topo canonique. Alors, si $A$ est local régulier et $J$ est son idéal maximal, le foncteur $\bRd j^!$ induit un foncteur
    $$
    \D^+_0(X, A) \to \D^+_0(Y, A).
    $$
    \item[(iv)] Sous les hypothèses préliminaires de (iii), le foncteur $\bRd j_*$ induit des foncteurs
    $$
    \D^-_0(Y, A) \to \D^-_0(X, A)
    $$
    $$
    \D^-_0(Y, A)_{\torf} \to \D^-_0(X, A)_{\torf}.
    $$
    Si de plus l'anneau $A$ est local régulier et $J$ est son idéal maximal, le foncteur $\bRd j_*$ induit un foncteur
    $$
    \D_0(Y, A) \to \D_0(X, A).
    $$
\end{itemize}
}
\vskip .3cm
{\bf Preuve} : Montrons par exemple (iii), les autres assertions se voyant de fa\c{c}on analogue. Étant données un entier $n \geq 0$ et $u: A \to A_n$ le morphisme d'anneaux canonique, il s'agit, d'après (8.1.16.(iii)), de prouver que $\bRd j^! (\bLd u^*(E)) \in \D^+_0(Y, A_n)$. Par hypothèse et d'après (8.2.6), il existe $F \in \D^+(A_n-\Mod_X)$ tel que $\bLd u^*(E) \isom \omega (F)$. On aura donc terminé si on prouve la commutativité du diagramme  
\[\begin{tikzcd}
	{\D^+(A_n-\Mod_X)} && {\D^+(X, A_n)} \\
	{\D^+(A_n-\Mod_Y)} && {\D^+(Y, A_n).}
	\arrow["\omega", from=1-1, to=1-3]
	\arrow["\omega", from=2-1, to=2-3]
	\arrow["{\bRd j^!}", from=1-3, to=2-3]
	\arrow["{\bRd j^!}"', from=1-1, to=2-1]
\end{tikzcd}\]
Or cela résulte immédiatement de ce que, comme on l'a déjà remarqué, un $A_n$--Module flasque définit un $A_n$-faisceau isomorphe à un $A_n$-faisceau flasque.



\vskip .3cm
{\bf 8.3. Parties multiplicatives}.
\vskip .3cm
{
Définition {\bf 8.3.1}. --- \it Étant donnés un topos $X$, un anneau $A$, un idéal $J$ de $A$ et une partie multiplicative $S$ de $A$, on appelle catégorie des $(A, J, S)$-faisceaux sur $X$ (ou, s'il n'y a pas de confusion possible, des $S^{-1}A$-faisceaux sur $X$), en on note $(A, J, S)-\fsc(X)$, ou plus simplement
$$
S^{-1}A-\fsc(X)
$$
la catégorie ainsi définie :
\begin{itemize}
    \item Ses objets sont les $A$-faisceaux. 
    \item Si $E$ et $F$ sont deux $A$-faisceau, un morphisme de $S^{-1}A$-faisceaux de $E$ dans $F$ est un élément de
    $$
    \Hom_{S^-1 A}(E, F) = S^{-1} \Hom_A (E, F).
    $$
\end{itemize}
}
\vskip .3cm
On vérifie facilement (cf. SGA5 VI 1.4.3.) que la catégorie $S^{-1}A-\fsc(X)$ est abélienne, et s'identifie de fa\c{c}on plus précise à la catégorie abélienne quotient de $A-\fsc(X)$ par la sous-catégorie abélienne épaisse engendrée par les $A$-faisceaux annulés par un élément de $S$.

Soient maintenant $B$ un autre anneau commutatif unifère, muni d'un idéal $K$, et $u: A \to B$ un morphisme d'anneaux vérifiant $u(J) \subset K$. Munissant $B$ de la partie multiplicative définie par $S$, il est clair que le foncteur $u_*: B-\fsc(X) \to A-\fsc(X)$ transforme un $B$-faisceau annulé par un élément de $S$ en un $A$-faisceau vérifiant la même propriété. Par suite, il induit par passage au quotient un foncteur exact, noté de même, 
$$
u_*: S^{-1} B-\fsc(X) \to S^{-1}A-\fsc(X),
$$
qu'on peut également définir directement grâce à la forme explicite des morphismes donnée en (8.3.1). Bien entendu, le diagramme
\[\begin{tikzcd}
	{B-\fsc(X)} && {A-\fsc(X)} \\
	{S^{-1}B-\fsc(X)} && {S^{-1}A-\fsc(X)}
	\arrow[from=1-1, to=2-1]
	\arrow[from=1-3, to=2-3]
	\arrow["{u_*}", from=2-1, to=2-3]
	\arrow["{u_*}", from=1-1, to=1-3]
\end{tikzcd}\leqno{(8.3.2)}\]
dans lequel les flèches verticales sont les morphismes de projection canoniques, est commutatif.
\vskip .3cm
{
Définition {\bf 8.3.3}. --- \it Supposons donnés $(X, A, J, S)$ comme plus haut. Avec les mêmes conventions simplificatrices que précédemment, on note
$$
\D^*(X, S^{-1}A) \quad (* = +, -, b, \emptyset)
$$
la catégorie ainsi définie :
\begin{itemize}
    \item Ses objets sont ceux de $\D^*(X, A)$.  
    \item Si $E$ et $F \in \D^*(X, A)$, $\Hom_{S^{-1} A}(E, F) = S^{-1} \Hom_A (E, F)$.
\end{itemize}
}
\vskip .3cm
Il est évident que la catégorie $\D^*(X, S^{-1}A)$ est additive, et que le foncteur translation de $\D^*(X, A)$ définit un foncteur de même type dans $\D^*(X, S^{-1} A)$, appelé également foncteur translation. De plus, la définition des morphismes de $\D^*(X, S^{-1}A)$ fournit de fa\c{c}on évidente un foncteur additif
$$
\D^*(X, A) \to D^*(X, S^{-1}A),
\leqno{(8.3.4)}
$$
qui induit l'identité sur les objets. Munissant alors $\D^*(X, S^{-1}A)$ des triangles qui sont isomorphes à l'image par (8.3.4) d'un triangle de $\D^*(X, A)$, et du foncteur translation précédemment défini, on constate aisément que l'on obtient ainsi une catégorie triangulée telle que le foncteur (8.3.4) soit exact. Plus précisément, on vérifie que $\D^*(X, A)$ en inversant les homothéties définies par des éléments de $S$.

Ces définitions étant ainsi posées, on laisse au lecteur le soin de s'assurer que le formalisme développé dans les numéros précédents pour les $A$-faisceaux s'étend, mutatis mutandis, aux $S^{-1}A$-faisceaux. 
