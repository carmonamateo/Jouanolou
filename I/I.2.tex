%%%%%%%%%%%%%%%%%%%%%%%%%%%%%%%%%%%%
\subsection*{2. Cas où l'objet final de $X$ est quasicompact.}
\addcontentsline{toc}{subsection}{2. Cas où l'objet final de $X$ est quasicompact}

On se propose maintenant de donner un certain nombre de catégories équivalentes à $A-\fsc(X)$, lorsque l'objet final de $X$ est quasicompact. Nous aurons besoin cela d'un certain nombre de lemmes techniques, dont la plupart n'utilisent pas cette hypothèse.

\vskip .3cm
{\bf 2.1}. Soit $(X, A, J)$ un idéotope. Étant donnés un objet $M$ de $\underline{\Hom}(N^\circ, A-\Mod_X)$, et une application croissante $\gamma \geq \id: \mathbf{N} \to \mathbf{N}$, on définit un nouveau système projectif $c_\gamma(M)$ en posant
$$
c_\gamma(M)_n = M_{\gamma(n)} \quad (n \geq 0),
$$
avec les morphismes de transition évidents. De plu, si $\gamma$ et $\delta$ sont deux applications de ce type, avec $\delta \geq \gamma$, on a un morphisme évident de systèmes projectifs
$$
c_\delta(M) \to c_\gamma(M).
$$
Ceci permet de définir une nouvelle catégorie, notée
$$
\underline{\Hom}_1(\mathbf{N}^\circ, A-\Mod_X), \quad \text{comme suit}.
$$
\begin{itemize}
    \item[(i)] Ses objets sont ceux de $\underline{\Hom}(N^\circ, A-\Mod_X)$. 
    \item[(ii)] Si $M$ et $M'$ sont deux objets de $\underline{\Hom}(N^\circ, A-\Mod_X)$, l'ensemble des morphismes de $M$ dans $M'$ est
    $$
    \Hom_1(M, M') = \varinjlim_\gamma \Hom(c_\gamma(M), M'),
    $$
    avec la loi de composition évidente.
\end{itemize}
Il est clair qu'un morphisme de $\underline{\Hom}(N^\circ, A-\Mod_X)$ définit un morphisme de $\underline{\Hom}_1(\mathbf{N}^\circ, A-\Mod_X)$, d'où un foncteur ``projection''
$$
q: \underline{\Hom}(N^\circ, A-\Mod_X) \to \underline{\Hom}_1(N^\circ, A-\Mod_X),
$$
qui est une bijection sur les objets.
\vskip .3cm
{\bf 2.2}. Soient maintenant $M$ un objet de $\mathcal{E}(X, J)$ et $\gamma \geq \id: \mathbf{N} \to \mathbf{N}$ une application croissante. On définit un nouvel objet $\chi_\gamma(M)$ de $\mathcal{E}(X, J)$, fonctoriel en $M$, par
$$
(\chi_\gamma (M))_n = M_{\gamma(n)}/J^{n+1}M_{\gamma(n)} \quad (n \geq 0),
$$
avec les morphismes de transition évidents. De plus, si $\delta$ est une application e même type, avec $\delta \geq \gamma$, on a un morphisme canonique
$$
\chi_\delta (M) \to \chi_\gamma (M).
$$
Ceci permet de définir une nouvelle catégorie $\mathcal{E}_1(X, J)$ comme suit.
\begin{itemize}
    \item[(i)] Ses objets sont ceux de $\mathcal{E}(X, J)$.
    \item[(ii)] Si $M$ et $N$ sont deux objets de $\mathcal{E}(X, J)$, l'ensemble des $\mathcal{E}_1(X, J)$-morphismes de $M$ dans $N$ est
    $$
    \Hom_1(M, N) = \varinjlim_\gamma \Hom(\chi_\gamma(M), N),
    $$
    avec la loi de composition évidente.
\end{itemize}
De même que précédemment, on a un foncteur projection
$$
q: \mathcal{E}(X, J) \to \mathcal{E}_1(X, J).
$$

Il est clair que la catégorie $\mathcal{E}_1(X, J)$ s'identifie à la sous-catégorie pleine de $\underline{\Hom}_1(\mathbf{N}^\circ, A-\Mod_X)$ engendrée pas les $A$-faisceaux, et que le diagramme
\[\begin{tikzcd}
	{\underline{\Hom}(\mathbf{N}^\circ, A-\Mod_X)} && {\underline{\Hom}_1(\mathbf{N}^\circ, A-\Mod_X)} \\
	{\mathcal{E}(X, J)} && {\mathcal{E}_1(X, J)}
	\arrow["q", from=1-1, to=1-3]
	\arrow["q", from=2-1, to=2-3]
	\arrow[hook', from=2-1, to=1-1]
	\arrow[hook', from=2-3, to=1-3]
\end{tikzcd}\]
dans lequel les flèches verticales sont les inclusions canoniques, est commutatif.

Enfin, on vérifie, comme dans (SGA5 V 2.4.1), que la famille des flèches canoniques du type $\chi_\gamma(M) \to M$ permet un calcul de fractions à droite, ce qui permet d'identifier $\mathcal{E}_1(X, J)$ à la catégorie obtenue à partir de $\mathcal{E}(X, J)$ en inversant ces flèches.
\vskip .3cm
{
Lemme {\bf 2.3}. --- \it La catégorie $\underline{\Hom}_1(\mathbf{N}^\circ, A-\Mod_X)$ (resp. $\mathcal{E}_1(X, J)$) est abélienne et le foncteur $q$ est exact.
}
\vskip .3cm
{\bf Preuve}: Montrons tout d'abord l'assertion non respée. Il est évident que la catégorie $\underline{\Hom}_1(\mathbf{N}^\circ, A-\Mod_X)$ est additive et que le foncteur $q$ ren inversibles toutes les flèches canoniques de la forme $c_\gamma(M) \to M$. Si $u: P \to Q$ est un élément de $\Hom_1(P, Q)$, alors $u$ est la classe d'une flèche $f: c_\gamma (P) \to Q$ et, en notant $c: c_\gamma (P) \to P$ la flèche canonique, on a donc $u = q(f) \circ q(c)^{-1}$ ; par suite, toute flèche de $\underline{\Hom}_1(\mathbf{N}^\circ, A-\Mod_X)$ est isomorphe à l'image par $q$ d'une flèche de $\underline{\Hom}(\mathbf{N}^\circ, A-\Mod_X)$. Il en résulte que pour voir que $\underline{\Hom}_1(\mathbf{N}^\circ, A-\Mod_X)$ est abélienne, il suffit de montrer que le foncteur $q$ est exact. Soient donc $P$ et
$$
0 \to M' \to M \to M'' \to 0
$$
respectivement un objet et une suite exacte de $\underline{\Hom}(\mathbf{N}^\circ, A-\Mod_X)$.

Comme les foncteurs $c_\gamma$ sont exacts et les limites inductives filtrantes de groupes abéliens sont exactes, les suites évidentes
$$
0 \to \varinjlim_\gamma \Hom(c_\gamma(P), M') \to \varinjlim_\gamma(c_\gamma(P), M) \to \varinjlim_\gamma \Hom(c_\gamma(P), M'')
$$
et
$$
0 \to \varinjlim_\gamma \Hom(c_\gamma(M''), P) \to \varinjlim_\gamma \Hom(c_\gamma(M), P) \to \varinjlim_\gamma \Hom(c_\gamma(M'), P)
$$
sont exactes, d'où l'assertion. L'assertion respée se voit de fa\c{c}on analogue; en fait, on montre en même temps que le foncteur d'inclusion
$$
\mathcal{E}_1(X, J) \to \underline{\Hom}_1(\mathbf{N}^\circ, A-\Mod_X)
$$
est exact
\vskip .3cm
{\bf 2.4}. Soient $P$ et $Q$ deux objets de $\underline{\Hom}(\mathbf{N}^\circ, A-\Mod_X)$, $\gamma \geq \id$ une application croissante de $\mathbf{N}$ dans $\mathbf{N}$ et $f: c_\gamma(P) \to Q$ un morphisme de systèmes projectifs. Désignant pour tout $n \in \mathbf{N}$ par $\xi_n$ la classe de $f_n$ dans $\varinjlim_m \Hom(P_m, Q_n)$, il est clair que l'ensemble des $\xi_n$ $(n \in \mathbf{N})$ définit un élément de 
$$
\varprojlim_n \varinjlim_m \Hom(P_m, Q_n),
$$
qui ne dépend que de la classe de $f$ dans $\Hom_1(P, Q)$. On définit ainsi un foncteur
$$
\rho: \underline{\Hom}_1(\mathbf{N}^\circ, A-\Mod_X) \to \Pro(A-\Mod_X),
$$
d'où un foncteur
$$
\delta: \mathcal{E}_1(X, J) \to \Pro(A-\Mod_X),
$$
obtenu par restriction de $\rho$ à $\mathcal{E}_1(X, J)$.
\vskip .3cm
{
Lemme {\bf 2.5}. --- \it Les foncteurs $\rho$ et $\delta$ ci-dessus sont des injections sur les objets et sont pleinement fidèles.
}
\vskip .3cm
{\bf Preuve}: Il suffit de le voir pour $\rho$. Dans ce cas, la seule assertion non tautologique est que $\rho$ est pleinement fidèle. Soient donc $P$ et $Q$ deux objets de $\underline{\Hom}(\mathbf{N}^\circ, A-\Mod_X)$, et montrons que l'application canonique
$$
\varinjlim_\gamma \Hom(c_\gamma(P), Q) \to \varprojlim_n \varinjlim_m \Hom(P_m, Q_n)
$$
est bijective.

{\bf Elle est injective}.  Soient $\gamma$ et $\gamma'$ deux applications croissantes $\geq \id$ de $\mathbf{N}$ dans $\mathbf{N}$, et
$$
f: c_\gamma(P) \to Q \quad \text{et} \quad f': c_{\gamma'}(P) \to Q
$$
deux morphismes ayant même image dans $\Pro \Hom(P, Q)$. Par hypothèse, pour tout entier $n \geq 0$, les morphismes $f_n$ et $f'_n$ définissent le même élément de $\varprojlim_m \Hom(P_m, Q_n)$, donc il existe un entier $\varphi(n) \geq \max(\gamma(n), \gamma'(n))$ tel que les composés 
$$
\begin{cases}
    P_{\varphi(n)} \xlongrightarrow{\can} P_{\gamma(n)} \xlongrightarrow{f_n} Q_n \\
    P_{\varphi(n)} \xlongrightarrow{\can} P_{\gamma'(n)} \xlongrightarrow{f'_n} Q_n
\end{cases}
$$
soient égaux. Il est alors immédiat que l'application $\gamma'': \mathbf{N} \to \mathbf{N}$ définie par 
$$
\gamma''(n) = \max_{n' \leq n}\delta(n')
$$
est croissante supérieure à l'identité, et que pour tout $n \geq 0$ les composés
$$
P_{\gamma''(n)} \xlongrightarrow{\can} P_{\gamma(n)} \xlongrightarrow{f_n} Q_n
$$
et
$$
P_{\gamma''(n)} \xlongrightarrow{\can} P_{\gamma'(n)} \xlongrightarrow{f'_n} Q_n
$$
sont égaux.

{\bf Elle est surjective}. Soient donnés
\begin{itemize}
    \item[a)] Une application $\theta: \mathbf{N} \to \mathbf{N}$.
    \item[b)] Une application $\lambda: \mathbf{N} \times \mathbf{N} \to \mathbf{N}$ vérifiant 
    $$
    \lambda(j, k) \geq \max (\theta(j), \theta(k))
    $$
    pour tout couple $(j, k)$.
    \item[c)] Pour tout entier $j$, un morphisme
    $$
    \xi_j: P_{\theta(j)} \to Q_j
    $$
    tel que, dès que $k \geq j$, le diagramme évident
    \[\begin{tikzcd}
	& {P_{\theta(j)}} && {Q_\gamma} \\
	{P_{\lambda(j, k)}} \\
	& {P_{\theta(k)}} && {Q_k}
	\arrow["\can"', from=3-4, to=1-4]
	\arrow["{\xi_k}"', from=3-2, to=3-4]
	\arrow["{\xi_j}", from=1-2, to=1-4]
	\arrow["\can", from=2-1, to=1-2]
	\arrow["\can"', from=2-1, to=3-2]
    \end{tikzcd}\]
    soit commutatif.
\end{itemize}
Il s'agit de trouver une application croissante $\gamma \geq \id$ de $\mathbf{N}$ dans $\mathbf{N}$ et un morphisme $f: c_\gamma(P) \to Q$ tel que pour tout $j \in \mathbf{N}$, les morphismes $f_j$ et $\xi_j$ aient même classe dans 
$$
\varprojlim_i \Hom(P_i, Q_j).
$$
On vérifie aisément que le couple $(\gamma, f)$ défini par 
$$
\begin{cases}
    \gamma(j) = \sup_{k, l \leq j} \lambda(k, l) \\ 
    f_j P_{\gamma(j)} \xlongrightarrow{\can} P_{\theta(j)} \xlongrightarrow{\xi_\gamma} Q_\gamma 
\end{cases}
$$
répond à la question.
\vskip .3cm
{\bf 2.6}. Supposons maintenant que l'objet final de $X$ soit quasicompact. Il est clair que le foncteur $q: \mathcal{E}(X, J) \to \mathcal{E}_1(X, J)$ envoie sur $O$ les $A$-faisceaux essentiellement nuls, ou, ce qui renvient au même négligeables. D'après la propriété universelle des catégories abéliennes quotients (Thèse Gabriel III 1 Cor.2), il admet donc une factorisation unique
\[\begin{tikzcd}
	{\mathcal{E}(X, J)} && {\mathcal{E}_1(X, J)} \\
	& {A-\fsc(X)}
	\arrow["q", from=1-1, to=1-3]
	\arrow["{\overline{q}}", from=1-3, to=2-2]
	\arrow["{\pi_X}"', from=1-1, to=2-2]
\end{tikzcd}\]
avec $\overline{q}$ un foncteur exact et $\pi_X$ le foncteur canonique de passage au quotient.
\vskip .3cm
{
Lemme {\bf 2.7}. --- \it Le foncteur $\overline{q}$ ci-dessus est un isomorphisme de catégories.
}
\vskip .3cm
{\bf Preuve}: Tenant compte de l'interprétation de $\mathcal{E}_1(X, J)$ comme catégorie de fractions (2.2), il s'agit de voir que si $C$ est une catégorie abélienne et
$$
F: \mathcal{E}(X, J) \to C
$$
un foncteur exact, les assertions suivantes sont équivalentes :
\begin{itemize}
    \item[(i)] $F$ annulle tout $A$-faisceau essentiellement nul. 
    \item[(ii)] $F$ rend inversible toute flèche canonique $\chi_\gamma(M) \to M$.
\end{itemize}
Si $M$ est un $A$-faisceau essentiellement nul, il existe une application croissante $\gamma \geq \id: \mathbf{N} \to \mathbf{N}$ telle que la flèche canonique $\chi_\gamma(M) \to M$ soit nulle, d'où (ii) $\Rightarrow$ (i). Prouvons (i) $\Rightarrow$ (ii).

Si $P$ et $Q$ sont définis par l'exactitude de la suite
$0 \to P \to \chi_\gamma(M) \xlongrightarrow{\can} M \to Q \to 0$, il est immédiat que les morphismes canoniques $\chi_\gamma(P) \to P$ et $\chi_\gamma(Q) \to Q$ sont nuls, d'où aussitôt l'assertion.
\vskip .3cm
{\bf Remarque 2.7.1}. Plus généralement, l'argument précédent montre, sans hypothèse sur $X$, qe $\mathcal{E}_1(X, J)$ est la catégorie abélienne quotient de $\mathcal{E}(X, J)$ par la sous-catégorie, abélienne engendrée par les $A$-faisceaux essentiellement nuls. On en déduit grâce à la description des morphismes d'une catégorie abélienne quotient, que tout morphisme (resp. isomorphisme) de $A-\fsc(X)$ est localement l'image d'un morphisme de $\mathcal{E}(X, J)$.
\vskip .3cm
{
Proposition {\bf 2.8}. --- \it Soit $(X, A, J)$ un idéotope. On suppose que l'objet final de $X$ est quasicompact. Alors le foncteur canonique
$$
\sigma \circ \overline{q}: A-\fsc(X) \to \Pro(A-\Mod_X)
$$
induit un isomorphisme de $A-\fsc(X)$ sur la sous-catégorie pleine de $\Pro(A-\Mod_X)$ engendrée par les $A$-faisceaux.
}
\vskip .3cm
{\bf Preuve}: Résulte immédiatement de (2.5) et (2.7).
\vskip .3cm
{\bf Remarque 2.8.1}. La proposition précédente s'applique notamment lorsque $X$ est le topos ponctuel.
\vskip .3cm
{\bf 2.8.2}. La preuve qu'on a donnée de (2.8) n'utilise pas le fait que la catégorie des pro-objets d'une catégorie abélienne, et il ne semble pas que l'utilisation de ce fait apporte des simplifications notables.

Soit $X$ un topos localement noethérien (SGA4 VI 2.11).

On rappelle que la sous-catégorie pleine de $X$ engendrée par les objets noethériens est stable par produits fibrés finis et que munissant $C$ de la structure de site induite par la topologie de $X$, le foncteur canonique
$$
X \to \widetilde{C}
$$
est une équivalence de catégories.
\vskip .3cm
{
Lemme {\bf 2.9}. --- \it Soient $(X, A, J)$ un idéotope, avec $X$ un topos localement noethérien, et $E$ et $F$ deux $A$-faisceaux sur $X$. La restriction au site des objets noethériens de $X$ du préfaisceau
$$
T \to \Hom(E|T, F|T)
$$
est un \emph{faisceau}.
}
\vskip .3cm
{\bf Preuve}: On est ramené à voir que si $T$ est un objet noethérien de $X$ et $(Y_i \to T)_{i \in I}$ est un recouvrement \emph{fini} de $T$ par des objets noethériens, alors la suite canonique
$$
0 \to \varinjlim_\gamma \Hom_a(\chi_\gamma (E)|T, F|T) \to \prod \varinjlim_\gamma \Hom_a(\chi_\gamma(E)|T, F|T),
$$
$$
\prod_i \varinjlim_\gamma \Hom_a(\chi_\gamma (E)| T_i \times_T T_j, F|T_i \times_T T_j)
$$
est exacte (2.7). Comme les produits finis sont des sommes directes et les limites inductives filtrantes de groupes abéliens sont exactes, l'assertion résulte de l'exactitude des suites canoniques 
$$
0 \to \Hom_a(\chi_\gamma (E) | T, F|T) \to \prod_i \Hom_a(\chi_\gamma(E)|T_i, F(T_i)) \to 
$$
$$
\to \prod_{i, j} \Hom_a(\chi_\gamma (E) | T_i \times_T T_j, F|T_i \times_T T_j).
$$
