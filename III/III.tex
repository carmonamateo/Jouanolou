%%%%%%%%%%%%%%%%%%%%%%%%%%%%%%%%%%%%%%%%%%%%%%%%%%%%%%%%%%%%%%%
\chapter*{\S \space III. --- APPLICATIONS AUX SCHÉMAS}\thispagestyle{empty}
\addcontentsline{toc}{section}{III. Applications aux schémas}
\label{ch:3}
\section*{}

Le texte qui suit ayant un caractère essentiellement provisoire (cf. l'appendice basé sur une construction 
de \emph{Deligne}), nous ferons toutes les hypothèses simplificatrices qui nous paraîtront nécessaires 
pour la clarté de l'exposé.

Soit $\ell$ un nombre premier. On fixe comme précédemment un anneau noethérien $A$ et un idéal propre $J$ 
de $A$. On suppose de plus que $A$ est une $\mathbf{Z}_{\ell}$-algèbre et que $J$ contient $\ell A$. 
Pour simplifier (cf. supra), tous les schémas considérés sont \emph{noethériens}.
